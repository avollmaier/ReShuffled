\lohead{Hörmann Stefan}
\chapter{Mechanik}

\section{Anforderung}
\label{sec:Anforderung}
Die Arbeit des mechanischen Teiles besteht darin, eine Maschine,
die Spielkarten mischen und ausgeben kann zu entwerfen, zu konstruieren
und einen Teilaufbau durchzuführen. Die Maschine sollte in der Lage
sein 20 Spielkarten zu Mischen und diese nach einem Spielmodus der
zuvor am LCD gewählt wurde auszugeben. Das Ziel ist es, die Spielkarten
optimal zu mischen, aber die Maschine dennoch kompakt und optisch
ansprechen zu entwerfen. Im weiteren sollte der Mischvorgang und dies
Ausgabe der Karten nicht zu lange dauern. Die Teile der Maschine
sollten so konstruiert werden, dass sie kostengünstig produziert
werden können. Zum Schluss sollte noch ein Teilaufbau der Maschine
geschehen, um die Funktionalität der einzelnen Bereiche zu testen
und gegeben falls zu verbessern.

\section{Problemstellungen}
Ein Problem ist das begrenzte Budget unseres Teams, somit sind
wir auf gewisse Produktionsarten unserer Bauteile beschränkt.
Dies hat zur folge, dass die Bauteile oft sehr simpel sind um
sie leichter zu konstruieren. Die Oberfläche der Karten ist ein
weiteres Problem, da diese nicht immer separieren lassen, dies
verursacht, dass oft zwei oder mehrere Karten auf einmal Genommen
werden und das Konzept des optimalen Mischens zerstört.

\section{Variantenvergleich}

\subsection{Anforderungen}

\begin{enumerate}
    \item Kosten  \\
    Der Automat sollte möglichst kostengünstig Produziert werden,
    da das vorhandene Budget gering ist. Dies hat zur Folge das keine teuren Motoren
    oder ähnliche Bauteile zum Einsatz kommen können und  keine teuren Bauteile produziert
    werden können.
    \item Schnelligkeit \\
    Um ein gutes Spielerlebnis zu garantieren sollte der Automat keine
    lange Mischzeit besitzen. Die Dauer in der man die Karten einführt und auf den
    Mischen-Button klickt bis hin zur Ausgabe der ersten Karte sollte möglichst gering sein.
    \item Mischgenauigkeit \\
    Die Mischgenauigkeit ist die am schwersten gewichtete Anforderung,
    da es das Ziel ist ein optimales Mischen der Spielkarten zu erreichen, sollte
    diese Anforderung mit größter Wichtigkeit erfüllt werden.
    \item Optik und Größe \\
    Die Optik des Automaten soll schlicht gehalten werden, jedoch sollte
    sie dennoch auf Messen und andere Ausstellungen präsentierbar sein. Der Automat
    sollte jedoch auch stabil konstruiert werden, muss aber dennoch mobil bleiben und
    darf eine gewisse Größe nicht Überschreiten.
\end{enumerate}

\subsection{Variantenvergleich}
Um alle oben angegebenen Anforderungen zu erfüllen, wurden mehrere Konzepte entworfen und diese verglichen.
\subsubsection{Variante 1 - Linearachsen}

Das erste Konzept würde mit zwei Linearachsen realisiert werden, diese wären im rechten Winkel zueinander
angeordnet. Die Senkrechte Linearachse ist mit einer Halterung versehen, diese Halterung ist in der Lage
ein Kartendeck aufzunehmen und die unterste Karte mithilfe eines Ausgaberades weiterzubefördern. Die zweite
Linearachse besitzt 4 Fächer in der die Karten von der ersten Linearachsenausgabe zufällig befördert werden.
Dies wird realisiert indem die erste Linearachse bei jeder Ausgabe zufällig das Fach durch hinauf und
hinabfahren wechselt. Befinden sich alle Karten im Lager, so fährt die erste Linearachse nach unten, danach
fährt die zweite Linearachse impulsiv nach links, um die Karten aus dem Lager zu befördern. Diese fallen
Senkrecht in das Lager der ersten Linearachse, wo sie nun zum Ausgeben durch das Ausgaberad bereit liegen.

\begin{wrapfigure}{r}{0.6\textwidth}
    \vspace{10pt}
    \begin{center}
        \includegraphics[width=0.55\textwidth]{Dateiname_Ohne_Endung}
    \end{center}
    \caption{Name des Bildes}
    \label{fig:verweis}
    \vspace{-10pt}
\end{wrapfigure}
\section{Konstruktion}

\section{Berechnungen, Festigkeitsanalyse und Dimensionierungen}

\section{Teilaufbau}
