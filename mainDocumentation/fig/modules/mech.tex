\lohead{Hörmann Stefan}
\chapter{Mechanik}

\section{Anforderung}
\label{sec:Anforderung}
Die Arbeit des mechanischen Teiles besteht darin, eine Maschine,
die Spielkarten mischen und ausgeben kann zu entwerfen, zu konstruieren
und einen Teilaufbau durchzuführen. Die Maschine sollte in der Lage
sein 20 Spielkarten zu Mischen und diese nach einem Spielmodus der
zuvor am LCD gewählt wurde auszugeben. Das Ziel ist es, die Spielkarten
optimal zu mischen, aber die Maschine dennoch kompakt und optisch
ansprechen zu entwerfen. Im weiteren sollte der Mischvorgang und dies
Ausgabe der Karten nicht zu lange dauern. Die Teile der Maschine
sollten so konstruiert werden, dass sie kostengünstig produziert
werden können. Zum Schluss sollte noch ein Teilaufbau der Maschine
geschehen, um die Funktionalität der einzelnen Bereiche zu testen
und gegeben falls zu verbessern.

\section{Problemstellungen}
Ein Problem ist das begrenzte Budget unseres Teams, somit sind
wir auf gewisse Produktionsarten unserer Bauteile beschränkt.
Dies hat zur folge, dass die Bauteile oft sehr simpel sind um
sie leichter zu konstruieren. Die Oberfläche der Karten ist ein
weiteres Problem, da diese nicht immer separieren lassen, dies
verursacht, dass oft zwei oder mehrere Karten auf einmal Genommen
werden und das Konzept des optimalen Mischens zerstört.

\section{Variantenvergleich}

\subsection{Anforderungen}

\begin{enumerate}
    \item Kosten
        \quad hallo
    \item Schnelligkeit
\end{enumerate}


\section{Konstruktion}

\section{Berechnungen, Festigkeitsanalyse und Dimensionierungen}

\section{Teilaufbau}
